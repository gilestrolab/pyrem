\begin{figure}[h!]
  \centering    
	\includegraphics[width=1.0\textwidth]{figures/error.pdf}
	\caption{\ctit{A posteriori confidence assessement.}
	\textbf{A}, Relation between the confidence value derived from proportions of votes (eq.\ref{eq:entropy}) and actual proportion of error.
	Cross-validation accuracy increases with empirical confidence value.
	For low values of confidence, $[0, 0.1]$, the predictor is not very reliable (more that $45\%$ error, while random errors would be $67\%$).
	In contrast, within the highest confidence range, $(0.9, 1.0]$, miscalssifications are very rare($0.5 \%$).
	\textbf{B}, Distribution of confidence values for all epochs. The overall median confidence value is at 0.69.
	\textbf{C}, Visualisation of approximatly 30 minutes of representative recording. 
	The doubt level ($1 - c$) can be displayed on top of the prediction annotations.
	In this example, the ground truth is also displayed to demonstrate examples of missclassification.
	\label{fig:error}
  }
\end{figure}
