\begin{figure}[h!]
  \centering    
    \includegraphics[width=1.0\textwidth]{figures/error.pdf}
    \caption{\ctit{A posteriori confidence assessment.}
    \textbf{A}, Relation between the confidence value derived from proportions of votes (eq.\ref{eq:entropy}) and actual proportion of error.
    Cross-validation error decreases with empirical confidence value.
    For low values of confidence, $[0, 0.1]$, the predictor is not very 
    reliable (more that $45\%$ error, whislt chance would be $67\%$).
    In contrast, within the highest confidence range, $(0.9, 1.0]$, misclassification are very rare($<0.5 \%$).
    \textbf{B}, Distribution of confidence values for all epochs. The overall median confidence value is at 0.69.
    \textbf{C}, Visualisation of approximately 30 minutes of representative \gls{eeg} and \gls{emg} recording.
    The variation in the most important \gls{eeg} variable, mean power in the cD\_6 sub-band is also plotted.
    Both predicted and ground truth states are represented by different colours (red: awake, blue: \gls{rem}, green: \gls{nrem}).
    The doubt level ($1 - c$) associated with predictions is displayed on top of the prediction annotations.
    \label{fig:error}
  }
\end{figure}
