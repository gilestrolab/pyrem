\section{Results} \label{results}

lorem ipsum ...

\subsection{DWT, feature extraction strategy}
Explain typycal approach, and why using DWT and then epoching is advantageaous while remaing fast.


\subsection{\texttt{Python} package}
Several algorithms to extract features from univariate time series had already been implemented in the \py{} package \pyeeg{}\citationneeded{}.
Unfortunately, some of them were critically slow, and could not realistically have been used in the present study.
Preliminary investigation of the source code revealed that runtimes may be improved by vectorising expressions and pre-allocating of temporary arrays.
Therefore, systematic reimplementation of all algorithms in \pyeeg{} was undertaken.
Very significant improvement in performance and scalability were achieved (table~\ref{tab:benchmark}).

Importantly, several mathematical inconsistancies between the original code and the mathematical definitions were also noticed.
This affected five of the eight reimplemented functions(table~\ref{tab:benchmark}). 
Detail of the corrections performed are provided, as notes, in the documentation of the new package\TODO{ref appendix}.
Numerical results for the three remaining functions were consitstant throughout optimisation.

In order to facilitate feature extrcation, several data structures and routines were also implemented 
in a new python package named \pr{}.
Briefly, extentions of \texttt{numpy} arrays providing metadata, sampling frequency, and other attributes were used to represent time series. 
User friencly indexing with string representing time was also developed.
In addition, a container for time series of discrete anotation levels, each linked to a confidence level, was built.
Importantly, a container for multiple time series, which supports different sampling frequencies
between time series was implemented.
The new package also provides visualisation, input/output, and wrappers for resampling and discrete wavelet decomposition.
Finally, unittests were implemented to ensure presistance of mathematical and programmatic validity though-out developmental stage.
A full documentation of \pr{} is provided in the appendix\TODO{ref} of the report herein.


\begin {table}[!h]
\begin{center}
\caption{\ctit{Performance improvements over \texttt{PyEEG}.}
In order to improve performance, modifications of the algorithms implemented in \texttt{PyEEG} were carried out.
This table compares how long, on average, each algorithm would take, for a random sequence of length $1280$ (\ie{} $5s$ at $256$Hz).
It also represents how many added points would lead to a tenfold runtime increase.
For the tested range ($n \in [1280;7680] $), all algorithms add approximately an
exponential time complexity ($10^{O(n)}$, $R^2 > 0.95$, for all).
Several mathematical inconsistencies were also discovered and corrected. 
The rightmost column (\textbf{\textdagger}) indicates whether the original implementation was
corrected in order to match mathematical definition. Each alteration is mathematically justified in the section \texttt{pyrem.univariate} of the \pr{} documentation (see appendix).
\textbf{(-)}: indicates a worse performance of \pr{} over \pyeeg{}.
Significance levels: $^{***}$, $p-value < 10^{-3}$; $^{**}$, $p-value < 10^{-2}$, see Material and Methods for detail about statistical analysis.
\label{tab:benchmark}
}
\footnotesize
\begin{tabular}{|c|c|c|c|c|c|c|}
  \hline
  &  & \multicolumn{2}{|c|}{\texttt{PyEEG}} & \multicolumn{2}{|c|}{\pr} & \\
 \hline
 \hline
 
  algorithm & function & \specialcell{$t$(ms) for \\$n = 1280$} & \specialcell{$n$ for $\times 10$\\increase} & \specialcell{$t$(ms) for \\$n = 1280$} & \specialcell{$n$ for $\times 10$\\ increase} & fix\textsuperscript{\textdagger}\\
 
  \hline
  \hline
\specialcell{Approximate\\Entropy} & \texttt{ap\_ent} &                                     9970 & 4288 & $487^{***}$ & $3478^{***}(-)$ & No\\
\hline
Fisher Information & \texttt{fisher\_info} &                                 3.24 & 8673 & $0.121^{***}$ & $12427^{***}$ & No\\
\hline
\specialcell{Higuchi\\Fractal Dimension} & \texttt{hfd} &                     11.7 & 8833 & $1.39^{***}$ & $28329^{***}$ & Yes\\
\hline
Hjorth parameters & \texttt{hjorth} &                                         5.14 & 8633 & $0.088^{***}$ & $36354^{***}$ & Yes\\
\hline
\specialcell{Petrosian\\Fractal Dimension} & \texttt{pfd} &                 2.66 & 8606 & 2.65 & 8579 & Yes\\
\hline
Sample Entropy & \texttt{samp\_ent} &                                         8305 & 4276 & $188^{***}$ & $5483^{***}(-)$ & No\\
\hline
Spectral Entropy & \texttt{spectral\_entropy} &                                 0.309 & 11459 & $0.227^{***}$ & $22133^{***}$ & Yes\\
\hline
\specialcell[l]{Singular Value \\Decomposition\\ entropy} & \texttt{svd\_ent} &     3.25 & 8663 & $0.113^{***}$ & $11774^{**}$ & Yes\\
 \hline
\end{tabular}
\end{center}
\end{table}



Interface imporvement (see package doc)

Visualisation (explain why it is important)


\subsection{Important features}
$n$ is large, reducing $p$ could make the analysis faster. computing each $p$ feature is slow.
variable importance can be used to select a subset of informative variable.

20ish variable are good enough.

The analysis can be rendered faster
\subsection{Including temporal information}
Using features at $\mathbf{Z} = \{\mathbf{X_{t-\tau}}, ..., \mathbf{X_{t}}, ..., \mathbf{X_{t+\tau}}\}$ provide a significant improvement over $\mathbf{Z} = \mathbf{X_{t}}$.

It does not seem adventageous to use $\tau > 3$.
\subsection{Deeper assesment}
